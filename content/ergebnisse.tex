\section{Ergebnisse}
\label{sec:Ergebnisse}

Nachdem beide Methoden trainiert wurden, wurden die so entstandenen Modelle evaluiert.
In \autoref{tab:accuracy} sind alle erreichten Genauigkeiten aufgelistet.

\begin{table}
    \centering
    \caption{Genauigkeit der verwendeten Methoden auf den verschiedenen Teildatensätzen, wobei mit CNN das Convolutional Neural Network gemeint ist}
    \label{tab:accuracy}
    \begin{tabular}{l c c c}
        \toprule 
        Methode & Trainingsdaten & Validierungsdaten & Testdaten \\ 
        \midrule 
        CNN (standard Schwellwert 0,50) & 93,56\% & 93,05\% & 93,30\% \\
        CNN (bester Schwellwert 0,54) & 93,58\% & 93,07\% & 93,32\% \\
        Random Forest & 89,40\% & 89,09\% & 89,37\% \\
        \bottomrule
    \end{tabular}
\end{table}

Hier ist deutlich zu erkennen, dass beide von uns gewählten Methoden gute Ergebnisse erzielt haben, jedoch die Hauptmethode eine deutlich höhere Genauigkeit erzielt.
Eine genauere Aussage über die Ergebnisse lässt sich über eine Konfusionsmatrix treffen. Diese ist in \autoref{fig:confusionmatrix} dargestellt.

\begin{figure}
    \centering
    \begin{subfigure}{0.4\textwidth}
        \centering
        \includegraphics[width=\textwidth]{images/cm_cnn.png}
        \caption{Hauptmethode}
        \label{fig:cm_cnn}
    \end{subfigure}
    \begin{subfigure}{0.4\textwidth}
        \centering
        \includegraphics[width=\textwidth]{images/cm_rndf.png}
        \caption{Alternativmethode}
        \label{fig:cm_rndf}
    \end{subfigure}
    \caption{Konfusionsmatrizen der Vorhersagen der verwendeten Methoden auf dem Testdatensatz}
    \label{fig:confusionmatrix}
\end{figure}

Um die Methoden allerdings tatsächlich evaluieren zu können ist es notwendig einige Beispiele der vorhergesagten Masken zu betrachten.
Dazu ist in \autoref{fig:beispiele_cnn} die Ausgabe des Convolutional Neural Networks mit und ohne binäre Darstellung mittels des Schwellwerts dargestellt.
Außerdem ist in \autoref{fig:beispiele_rndf} die zusätzliche Eingabe des Gradienten des Satellitenbildes und die Ausgabe des Random Forest zu sehen.

\begin{figure}
    \centering
    \includegraphics[width=0.7\textwidth]{images/bsp_cnn.png}
    \caption{Beispiele der Vorhersagen des Convolutional Neural Network. %
    Aufgezeigt ist zu einem Satellitenbild die Maske aus dem Datensatz und %
    die vorhergesagte Maske mit und ohne binärisierung über den Schwellwert. %
    Außerdem sind die zugehörigen Länder aufgelistet.\\ \copyright Mapbox, \copyright OpenStreetMap}
    \label{fig:beispiele_cnn}
\end{figure}

\begin{figure}
    \centering
    \includegraphics[width=0.7\textwidth]{images/bsp_rndf.png}
    \caption{Beispiele der Vorhersagen des Random Forest. %
    Aufgezeigt ist zu einem Satellitenbild das Bild der Gradienten, %
    die Maske aus dem Datensatz und %
    die vorhergesagte Maske. %
    Außerdem sind die zugehörigen Länder aufgelistet.\\ \copyright Mapbox, \copyright OpenStreetMap}
    \label{fig:beispiele_rndf}
\end{figure}

Schon hier lässt sich erkennen, dass die Alternativmethode deutlich mehr Rauschen aufgrund der pixelweisen Klassifizierung hat.
Die Hauptmethode kann nicht alle Pixel eindeutig zuordnen und vor Allem Ränder oder schwer zu klassifizierende Stellen erhalten Werte die eher mittig zwischen 0 und 1 liegen.
Auch ist schon zu sehen, dass die gegebene Maske nicht perfekt ist und teilweise Gewässer gar nicht oder nur grob markiert.

Um weitere Erkenntnisse zu den beiden Methoden zu bekommen, werden in \autoref{fig:vergleich} die Vorhersagen beider Methoden miteinander verglichen.
Weitere Beispiele sind im Anhang in \autoref{fig:bsp} und in \autoref{fig:bsp_bad} dargestellt.

\begin{figure}
    \centering
    \includegraphics[width=0.7\textwidth]{images/vergleich.png}
    \caption{Beispiele zum Vergleich der Hauptmethode zur Alternativmethode.\\ \copyright Mapbox, \copyright OpenStreetMap}
    \label{fig:vergleich}
\end{figure}

Häufig ist die gegebene Maske nicht sehr detailliert 
und obwohl beide Modelle auf diesen fehlerhaften Daten trainiert wurden,
werden die Grenzen der Gewässer häufig klar erkannt.
Aufgrund dieser fehlerhaften gegebenen Masken war auch eine Genauigkeit von 100\% nicht erreichbar.

In \autoref{fig:bsp_bad} sind zusätzlich noch einige Beispiele aufgezeigt, 
bei denen die Segmentierung nicht gut funktioniert hat 
oder die gegebene Maske auch falsch ist.
Probleme sind z.B. Wolken, Ebbe und Flut, Felder, Sumpfgebiete oder sonstige auch für den Menschen schwer zu erkennende Gewässer.
Um tatsächlich alle Probleme sehen zu können, müssten deutlich mehr Bilder gezeigt werden, dafür ist allerdings in diesem Bericht kein Platz.

Teilweise ist nicht erklärbar warum die Algorithmen Wasser falsch erkennen.
Häufig entstehen Fehler an Stellen, die auch ein Mensch nicht erkennen könnte, aber manchmal werden eindeutige Kanten nicht erkannt.
Einige der Fehler sind sicherlich auf die niedrige Auflösung von 128x128 Pixel zurückzuführen, aber die meisten Gewässer waren auch mit dieser Auflösung zu erkennen.
