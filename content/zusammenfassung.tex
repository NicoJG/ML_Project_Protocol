\section{Fazit}
\label{sec:Fazit}

Die Ergebnisse beider verwendeten Methoden zur Segmentierung von Gewässern auf Satellitenbildern waren ausreichend
und übertrafen teilweise unsere Erwartungen vor dem Projekt.
Erst während des Projektes erkannten wir, dass einige der Masken im Datensatz fehlerhaft waren.
Auch erkannten wir, dass ein paar der Satellitenbilder für eine Segmentierung unbrauchbar waren. (z.B. Bilder von Wolken)
Trotz der schlechten Daten, die zum Training verwendet wurden, wurden gute Ergebnisse erzielt 
und die vorhergesagten Masken waren häufig besser als die Masken des Datensatzes.
Die erzielten Genauigkeiten der Hauptmethode von etwa 93\% und der Alternativmethode von etwa 89\% lassen sich allerdings nicht näher beurteilen, 
da nicht klar ist welche Genauigkeit die Masken im Datensatz erreichen und diese Masken als Grundwahrheit angenommen wurden.
Hierfür wäre eine manuelle Überprüfung notwendig.

Der Vergleich des Convolutional Neural Network zum Random Forest zeigte deutlich, dass für dieses Problem ersteres besser geeignet war.
Zusätzlich zu den besseren Ergebnissen war das Convolutional Neural Network deutlich sparsamer mit der benötigten Rechenleistung 
und sogar die Vorhersagen wurden deutlich schneller erzeugt als bei dem Random Forest.
Auch ist zu sehen, dass das Convolutional Neural Network Beziehungen zwischen den Pixeln auswertete, was bei dem Random Forest nicht möglich war.

Durch eine ausführlichere Hyperparametersuche könnte das Modell weiter optimiert werden. 
Es wurde allerdings gezeigt, dass ein so komplexes und vielseitiges Problem 
wie die Wassererkennung auf Satellitenbildern mit Deep Learning gelöst werden kann.
Eine Überwachung der Ausgabe von neuronalen Netzen durch Menschen ist allerdings noch zu empfehlen.

Abschließend sei zu sagen, dass dieses Projekt uns den Umgang mit Deep Learning weiter nahegebracht hat 
und gezeigt hat wie vielseitig einsetzbar neuronale Netzwerke sind.
