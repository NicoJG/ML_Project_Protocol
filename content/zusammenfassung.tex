\section{Zusammenfassung}
\label{sec:Zusammenfassung}

Die Ergebnisse beider verwendeten Methoden zur Segmentierung von Gewässern auf Satellitenbildern waren ausreichend
und übertrafen teilweise unsere Erwartungen vor dem Projekt.
Erst während dem Projekt erkannten wir, dass einige der Masken im Datensatz fehlerhaft sind.
Auch erkannten wir, dass ein paar der Satellitenbilder für eine Segmentierung unbrauchbar waren. (z.B. Bilder von Wolken)
Trotz der schlechten Daten, die zum Training benutzt wurden, wurden gute Ergebnisse erzielt 
und die vorhergesagten Masken waren häufig besser als die Masken des Datensatzes.
Die erzielten Genauigkeiten der Hauptmethode von etwa 93\% und der Alternativmethode von etwa 89\% lassen sich allerdings nicht näher beurteilen, 
da nicht klar ist welche Genauigkeit die Masken im Datensatz erreichen.
Hierfür wäre eine Manuelle Überprüfung notwendig.

Der Vergleich des Convolutional Neural Network zum Random Forest zeigt deulich, dass für dieses Problem ersteres besser geeignet war.
Zusätzlich zu den besseren Ergebnissen war das Convolutional Neural Network deutlich sparsamer mit der benötigten Rechenleistung 
und sogar die Vorhersagen gingen deutlich schneller als bei dem Random Forest.
Auch ist zu sehen, dass ein Convolutional Neural Network Beziehungen zwischen den Pixeln auswertet, was bei einem Random Forest nicht möglich ist.

Sicherlich ist das verwendete Modell nicht perfekt, aber es zeigt, dass ein so komplexes und vielseitiges Problem 
wie die Wassererkennung auf Satellitenbildern mit Deep Learning gelöst werden kann.
Eine Überwachung der Ausgabe von Neuronalen Netzen durch Menschen ist allerdings noch zu empfehlen.

Abschließend sei zu sagen, dass dieses Projekt uns den Umgang mit Deep Learning weiter nahegebracht hat 
und gezeigt hat wie vielseitig einsetzbar Neuronale Netzwerke sind.
