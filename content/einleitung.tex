\section{Einleitung und Zielsetzung}
\label{sec:Einleitung}

Schon lange ist die Kartografie ein wesentlicher Teil des menschlichen Fortschritts.
Gegen Ende des 20. Jahrhunderts gewann die Kartografie anhand von Satellitenbildern zunehmend an Wichtigkeit.
Dadurch hatten und haben Kartografen einen deutlich geringeren Aufwand große Gebiete der Erde abzudecken.
Da allerdings die zu kartografierende Fläche auf der Erde sehr groß ist, scheint eine vollständig manuelle Kartografie nicht mehr sinnvoll zu sein.
Ein zusätzliches Problem ist, dass Karten durchgehend angepasst werden müssen, da die Welt kontinuierlich im Wandel ist, 
sowohl durch menschliches Einwirken als auch durch natürliche Phänomene.

Um das Erstellen von Karten deutlich zu beschleunigen, wird seit einigen Jahren die Analyse der Satellitenbilder hauptsächlich von Computern durchgeführt.
Hierfür eignet sich maschinelles Lernen und insbesondere das \enquote{Deep Learning} mit neuronalen Netzwerken.
Aber auch wenn heutzutage große Konzerne wie Google, Apple oder Microsoft schon sehr gute Algorithmen entwickelt haben,
ist es immer noch deutlich zu erkennen, dass auch diese Algorithmen noch Fehler machen.

\textbf{Fragestellung:} Wo befinden sich auf einem gegebenen Satellitenbild innerhalb Europas größere Wasserflächen?

Ziel dieses Projektes ist also die Erkennung von Gewässern auf Satellitenbildern.
Genauer gesagt wird mithilfe von maschinellem Lernen zu jedem Eingangsbild eine Maske erzeugt, die jedem Pixel entweder Wasser oder kein Wasser zuordnet.
Außerdem werden nur Satellitenbilder innerhalb europäischer Länder und auf einer bestimmten Zoomstufe betrachtet.
Dadurch können nur größere Gewässer, also Seen, Flüsse, das Meer oder ähnliches erkannt werden und kleinere Gewässer wie Bäche, Teiche oder ähnliches sind nicht von großer Bedeutung für dieses Projekt.
Weitere Details zum Datensatz sind in \autoref{sec:Datensatz} erläutert.

Das Projekt ist Teil des Seminars \enquote{Maschinelles Lernen für Physiker*innen} im Sommersemester 2021 und wurde von Samuel Haefs und mir (Nico Guth) durchgeführt.
